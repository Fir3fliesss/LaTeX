\documentclass[a4paper, 14pt]{extarticle}
\usepackage[a4paper, margin=1in]{geometry}
\usepackage{newtxtext,newtxmath}
\usepackage{setspace}
\usepackage{ragged2e}
\usepackage{indentfirst}

\renewcommand{\normalsize}{\fontsize{12pt}{14pt}\selectfont}

\titleformat{\section}[block]{\centering\bfseries\fontsize{14pt}{16pt}\selectfont}{}{0pt}{}


\onehalfspacing
\setlength{\parindent}{2em}

\begin{document}

\section*{General Classification}

\justifying
Creative workers are employees who have high abilities in creating unique and cool works. They usually work in creative industries and earn money from creative based activities. They are also called creative specialists and are hired because of their expertise in the creative field.

\bigskip

\section*{General Classification}

There are various types of creative workers, such as architects, actors, content creators, scriptwriters, photographers, videographers, and more. These workers use their creativity to make money and often work in different places according to their respective provisions.

Being a creative worker is also not always fun. There are several problems that they often face as creative workers. For example, they do not have a definite workplace like office workers. They also have difficulty managing time because their working hours are erratic; they can work at any time according to circumstances. In addition, creative workers also experience many revisions of their work.

For example, a freelancer such as a video editor must always be available and often gets revisions because the results are not what the client wants. The characteristic feature of creative workers is that they always rely on creativity and are usually more independent. Some creative workers, such as photographers, must always have tools available for certain moments and have high creativity so they can look at things from a different perspective than usual.

Even so, being a creative worker is worth it. Why? Of course, because the advantages of utilizing creativity are enormous. They can increase the price of a product. For example, a product can have more value just because it is presented in a different place and with different packaging. Another example, creative workers can process something to become more valuable, such as plastic packaging into bags, leftover cloth into wallets, coconut shells into piggy banks, used bottles into flowers and vases, and many more. 

Apart from raising prices and processing goods, another advantage of creative workers is being able to stand out from their competitors so that the products they sell are more attractive to consumers. Not only that, but being a creative worker can also make it easier for someone, for example, by collaborating their creativity with technology, such as store sellers who can sell their products online.

We also made observations with one of the creative workers, namely a ramen seller on Jl. Pahlawan, and these are the results of our observations. The first thing we discovered was their uniqueness. They sell products that have absolutely no competitors, so they can dominate the market easily. Another unique thing that we found is that the place is very nice and comfortable. They created a place with a different vibe, and it's amazing. They also create unique blends, which makes them look like a cafe but sell ramen.

That is the power of creative workers. In this era of globalization, all jobs require creativity to compete with other competitors. It can be concluded that being a creative worker is a very promising job.

\end{document}